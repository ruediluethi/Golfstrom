\documentclass[a4paper,twoside]{article}

\usepackage[utf8]{inputenc}
\usepackage[ngerman]{babel}

\usepackage{epsfig}
\usepackage{subfigure}
\usepackage{calc}
\usepackage{amssymb}
\usepackage{amstext}
\usepackage{amsmath}
\usepackage{amsthm}
\usepackage{multicol}
\usepackage{pslatex}
\usepackage{apalike}
\usepackage{enumitem}
\usepackage{MOSI}     % Please add other packages that you may need BEFORE the MOSI.sty package.

\newcommand{\team}{Patrick Neher, Ruedi Lüthi}
\newcommand{\theme}{Golfstrom}
\setcounter{page}{1}

\subfigtopskip=0pt
\subfigcapskip=0pt
\subfigbottomskip=0pt

\begin{document}

	\title{Der Golfstrom\subtitle{ein einfaches mathematisches Modell} }
	
	\author{\authorname{Patrick Neher und Ruedi Lüthi}}
	
	\keywords{Golfstorm, Stabilitätsanalyse}

	\abstract{Bla bla bla}
	
	\onecolumn \maketitle \normalsize \vfill

	\section{\uppercase{Der Golfstrom}}\label{sec:Golfstrom}

	\noindent Der Golfstrom ist ein Naturphänomen, welches bla bla bla....
	\begin{verbatim}
	 	https://www.wetter.com/news/golfstrom-die-heizung-europas_aid_55c36ad3cebfc0cc4d8b464b.html
	 	http://www.naplesgolf.org/napl20_2.htm
	 	https://meteo.plus/wassertemperaturen-norwegen-west.php
	 	http://www.travelklima.de/nordmeer-kreuzfahrten/
	\end{verbatim}
	
	\section{\uppercase{Das Modell}}\label{sec:Modell}
	
	\noindent Behälter und so weiter
	
	\subsection{Entdimensionalisierung}
	
	\subsection{Gleichgewichtsuntersuchung}
	
	\section{\uppercase{Anwendung}}\label{sec:Anwendung}
	
	\subsection{Datenabschätzung}
	\noindent Für Temperatur:
	% da sollte es noch was besseres geben um Links einbetten zu können?
	\begin{verbatim}
		https://www.seatemperature.org/north-sea
		https://www.seatemperature.org/mexico-sea
	\end{verbatim}
	Aber bei dieser Quelle leider keine Monatsdaten... \\
	\\
	Etwas zum Salzgehalt:
	\begin{verbatim}
		https://en.wikipedia.org/wiki/North_Sea#Temperature_and_salinity
	\end{verbatim}
	aber halt leider nur zur Nordsee \\	
	\\
	Sowas wär halt toll als Quelle:
	\begin{verbatim}
		https://data.giss.nasa.gov/
	\end{verbatim}
	Aber habe da bisher nur Daten zur Oberfächentemperatur gefunden... \\
	\\
	Vielleicht findet man da auch noch was, wenn man die Quellen durchgeht?
	\begin{verbatim}
		https://en.wikipedia.org/wiki/Sea_surface_temperature
	\end{verbatim}
	
	
	\subsection{Temperaturschwankungen durch Jahreszeiten}
	
	\subsection{Die Arktis schmilz}
	
	\section{\uppercase{Fazit}}\label{sec:Fazit}

\end{document}	

