\documentclass[11pt,a4paper]{article} 
\usepackage[utf8]{inputenc}
\usepackage[ngerman]{babel}
\usepackage{amsmath}
\usepackage{amssymb}
\usepackage{geometry}
\usepackage{hyperref}
\usepackage[table]{xcolor}
\geometry{
  left=3cm,
  right=3cm,
  top=3cm,
  bottom=3cm,
  bindingoffset=0mm
}
\usepackage{graphicx}
\usepackage{dsfont}

\begin{document}

	Ursprüngliche Gleichungen aus dem Konvektionsmodell:
	\begin{align*}
		\frac{dT^*_1}{dt^*} &= k_T\left(T^*_{01} - T^*_1\right) + \left|q^*\right|\left(T^*_2 - T^*_1\right) \\
		\frac{dT^*_2}{dt^*} &= k_T\left(T^*_{02} - T^*_2\right) + \left|q^*\right|\left(T^*_1 - T^*_2\right) \\
		\frac{dS^*_1}{dt^*} &= k_S\left(S^*_{01} - S^*_1\right) + \left|q^*\right|\left(S^*_2 - S^*_1\right) \\
		\frac{dS^*_2}{dt^*} &= k_S\left(S^*_{02} - S^*_2\right) + \left|q^*\right|\left(S^*_1 - S^*_2\right)
	\end{align*}
	
	Reduzieren der Gleichungen, indem nur die Differenz betrachtet wird:
	\begin{align*}
		T^* = T^*_1 - T^*_2 &\Rightarrow \left\{
		\begin{array}{l}
			T^*_1 = T^* + T^*_2 \\
			T^*_2 = T^*_1 - T^*
		\end{array} \right. \\
		\frac{dT^*}{dt^*} = \frac{dT^*_1}{dt^*} - \frac{dT^*_2}{dt^*} &= 
		k_T\left(T^*_{0} + T^*_{02} - T^* - T^*_2 \right) + \left|q^*\right|\left(T^*_2 - T^*_1\right)  - \left( k_T\left(T^*_{02} - T^*_2\right) + \left|q^*\right|\left(T^*_1 - T^*_2\right) \right) \\
		&= k_T\left(T^*_{0} - T^*\right) + k_T\left(T^*_{02} - T^*_2\right) + \left|q^*\right|\left(T^*_2 - T^*_1\right) - k_T\left(T^*_{02} - T^*_2\right) - \left|q^*\right|\left(T^*_1 - T^*_2\right) \\
		&= k_T\left(T^*_{0} - T^*\right) - 2\left|q^*\right|\left(T^*_1 - T^*_2\right) \\
		&= k_T\left(T^*_{0} - T^*\right) - 2\left|q^*\right|T^*
	\end{align*}
	
	Der Fluss berechnet sich ja gerade jeweils durch die Differenzen der beiden Behälter:
	\begin{align*}
		S^* = S^*_1 - S^*_2 &\Rightarrow \left\{
		\begin{array}{l}
			S^*_1 = S^* + S^*_2 \\
			S^*_2 = S^*_1 - S^*
		\end{array} \right. \\
		\frac{dS^*}{dt^*} &= k_S\left(S^*_{0} - S^*\right) - 2\left|q^*\right|S^*
	\end{align*}
	
	\begin{align*}
		q^* = a\left( \rho^*_2 - \rho^*_1 \right) &=
		a \left( 
			\rho^*_0 - bT^*_2 + cS^*_2 -
			\left( \rho^*_0 - bT^*_1 + cS^*_1 \right)
		\right) \\
		&= a \left( b \left(T^*_1 - T^*_2\right) + c \left( S^*_2 - S^*_1 \right) \right) \\
		&= a \left( bT^* - cS^* \right)
	\end{align*}
	
	\newpage	
	
	Entdimensionalisierung:	
	
	\begin{align*}
		t = k_T \cdot t^* \Rightarrow t^* = \frac{t}{k_T} &\qquad
		q = 2\frac{q^*}{k_T} \Rightarrow q^* = \frac{q \cdot k_T}{2} \\
		T = \frac{T^*}{T^*_0} \Rightarrow T^* = T \cdot T^*_0 \\
		\frac{dT^*}{dt^*} &= k_T\left(T^*_{0} - T^*\right) - 2\left|q^*\right|T^* \\
		\frac{dT \cdot T^*_0}{\frac{dt}{k_T}} &= 
		k_T\left(T^*_{0} - T \cdot T^*_0 \right) - 2\left|\frac{q \cdot k_T}{2}\right|T \cdot T^*_0 \\
		\stackrel{\substack{
			T^*_{0} \textrm{ konst.}\\
			k_T > 0
		}}{\Rightarrow} \frac{dT}{dt} &=
		\left(1 - T\right) - \left|q\right|T \\
		S = \frac{S^*}{S^*_0} \Rightarrow S^* = S \cdot S^*_0 \\
		\frac{dS^*}{dt^*} &= k_S\left(S^*_{0} - S^*\right) - 2\left|q^*\right|S^* \\
		\frac{dS \cdot S^*_0}{\frac{dt}{k_T}} &= k_S\left(S^*_{0} - S \cdot S^*_0\right) - 2\left|\frac{q \cdot k_T}{2}\right|S \cdot S^*_0 \\
		\Rightarrow \frac{dS}{dt} &= \underbrace{\frac{k_S}{k_T}}_{=\gamma}\left(1 - S\right) - \left|q\right|S
	\end{align*}
	\(\gamma :\) beschreibt das Verhältnis des Einflusses von Umgebungstemperatur zu Umgebungssalzgehalt. \( \rightarrow \) Was vermischt sich schneller? Die Temperatur oder der Salzgehalt?
	
	\begin{align*}
		q^* &= a \left( bT^* - cS^* \right)
		\Rightarrow \frac{q \cdot k_T}{2} = a\left(b \cdot T \cdot T^*_0 - c \cdot S \cdot S^*_0 \right) \\
		\Rightarrow q &= \underbrace{2\frac{ab}{k_T}T^*_0}_{=\alpha}T - \underbrace{2\frac{ac}{k_T}S^*_0}_{=\beta}S
	\end{align*}
	
	\newpage	
	
	Gleichgewichtsuntersuchung:
	\begin{align*}
		\frac{dT}{dt} &= \left(1 - T\right) - \left|q\right|T \stackrel{!}{=} 0 \\
		&\Rightarrow 1 - T\left(1 + |q|\right) \stackrel{!}{=} 0 \\
		&\Rightarrow T = \frac{1}{1+|q|} \\
		\Rightarrow \frac{dS}{dt} &= \gamma \left(1 - S\right) - \left|q\right|S \stackrel{!}{=} 0 \\
		&\Rightarrow S = \frac{\gamma}{\gamma + |q|}
	\end{align*}
	
	Flussmenge wenn \(S\) und \(T\) im Gleichgewicht:
	\begin{align*}
		q = \alpha T - \beta S
		 = \alpha \frac{1}{1+|q|} - \beta \frac{\gamma}{\gamma + |q|} = g(q)
	\end{align*}
	
	Sei nun (Schnittpunkte der Winkelhalbierenden mit g(q))
	\begin{align*}
		q = g(q) &\Rightarrow 0 = q - g(q) \stackrel{q > 0}{=}  q - \left(\alpha \frac{1}{1+q} - \beta \frac{\gamma}{\gamma + q} \right) \\
		& \stackrel{\substack{\textrm{warum}\\\textrm{äquivalent?}}}{\Rightarrow}
		k(q) := \left( q - g(q) \right) (1+q) (\gamma + q) = 0 \\
		k(q) &= q(1+q) (\gamma + q) - \alpha \frac{(1+q) (\gamma + q)}{1+q} + \beta \frac{\gamma (1+q) (\gamma + q)}{\gamma + q} \\
		&= q(1+q) (\gamma + q) - \alpha (\gamma + q) + \beta \gamma (1+q) \\
		&= q \gamma + q^2 + q^2 \gamma + q^3 - \alpha \gamma - \alpha q + \beta \gamma + \beta \gamma q \\
		&= q^3 + q^2(1 + \gamma) + q(\gamma - \alpha + \beta \gamma) - \alpha \gamma + \beta \gamma \\
		&= q^3 + q^2(1 + \gamma) + q\left(\gamma \left(1 + \beta\right) - \alpha \right) + \gamma \left( \beta - \alpha \right) \\
		& \stackrel{\textrm{Nullstellen}}{\Rightarrow} k(q) \stackrel{!}{=} 0 \\
		k(0) &= \gamma \left( \beta - \alpha \right)
	\end{align*}
	
	\begin{align*}
		k(q) &= \left( g\left(q\right) - q \right)(1 - q)(\gamma - q) \\
		&= -q(1 - q) (\gamma - q) + \alpha \frac{(1 - q) (\gamma - q)}{1 - q} - \beta \frac{\gamma (1 - q) (\gamma - q)}{\gamma - q} \\
		&= -q(1 - q)(\gamma - q) + \alpha(\gamma - q) - \beta\gamma(1 - q) \\
		&= -q(\gamma - q - q\gamma + q^2) + \alpha\gamma - \alpha q - \beta\gamma + \beta \gamma q \\
		&= -\gamma q + q^2 + \gamma q^2 - q^3 + \alpha\gamma - \alpha q - \beta\gamma + \beta \gamma q \\
		&= -q^3 + q^2(1 + \gamma) + q(-\gamma -\alpha + \beta \gamma) + \gamma(\alpha - \beta) \\
		&= -q^3 + q^2(1 + \gamma) + q(\gamma(\beta - 1) -\alpha) + \gamma(\alpha - \beta) \\
		k'(q) &= -3q^2 + 2q(1+\gamma) + \gamma(\beta - 1) -\alpha \\
		k(0) &=  \gamma(\alpha - \beta)
	\end{align*}		
	
	\begin{align*}
		q_w &= \frac{1}{3} \left( -\sqrt{-3\alpha + 3\beta\gamma + \gamma^2 - \gamma + 1} + \gamma + 1 \right) \\
		&= \frac{1}{3} \left( -\sqrt{-3\alpha + \gamma(3\beta + \gamma - 1) + 1 } + \gamma + 1 \right)
	\end{align*}
	
	Da wir jeweils nur die Seite \(q > 0\) betrachten, können genauso gut auch folgende Funktion betrachten:
	\begin{align*}
		\tilde{g}(q) = q - \alpha \frac{1}{1+q} - \beta \frac{\gamma}{\gamma + q}
	\end{align*}
	Nullstellen von \(q - \tilde{g}(q) \stackrel{!}{=} 0\):
	
	\newpage
	\begin{align*}
		\tilde{q} &= \alpha \frac{1}{1+ |\tilde{q}| } - \beta \frac{\gamma}{\gamma + |\tilde{q}| } = \alpha \tilde{T} - \beta \tilde{S} \\
		\frac{dT}{dt} &= (1 - T) - |q(T,S)|T \\
		\frac{dS}{dt} &= \gamma(1 - S) - |q(T,S)|S \\
	\end{align*}		
	
\end{document}