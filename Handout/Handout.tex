\documentclass[a4paper,twoside]{article}

\usepackage[utf8]{inputenc}
\usepackage[ngerman]{babel}
\usepackage{url} %Bib
\usepackage[sort&compress,numbers]{natbib} %Bib
\usepackage{fullpage}
\usepackage{lipsum}
\usepackage{float}
\usepackage{cuted}
\usepackage{epsfig}
\usepackage{subfigure}
\usepackage{calc}
\usepackage{amssymb}
\usepackage{amstext}
\usepackage{amsmath}
\usepackage{amsthm}
\usepackage{multicol}
\usepackage{pslatex}
\usepackage{apalike}
\usepackage{enumitem}
\usepackage{bbm}
\usepackage{etoolbox}
\usepackage{textcomp}
%\usepackage{hyperref}
\apptocmd{\UrlBreaks}{\do\f\do\m}{}{}
\usepackage{MOSI}     % Please add other packages that you may need BEFORE the MOSI.sty package.

\newcommand{\team}{Patrick Neher, Ruedi Lüthi}
\newcommand{\theme}{Golfstrom}
\setcounter{page}{1}

\subfigtopskip=0pt
\subfigcapskip=0pt
\subfigbottomskip=0pt

\begin{document}
	
	\begin{center}
	\begin{LARGE}
			\uppercase{\textbf{Der Golfstrom}} \\[12pt] 
	\end{LARGE}
	\end{center}

	
	
	Dieses Handout dient dazu die Rechnung, während der Präsentation besser zu verstehen.
	
	\section*{\uppercase{Zur Differenz}\label{sec: Differenz}}
	
	\begin{footnotesize}
	\noindent Das sind die 4 Gleichung von unserer Modellskizze:
	\begin{align*}
		\frac{dT^*_1}{dt^*} &= k_T\left(T^*_{01} - T^*_1\right) + \left|q^*\right|\left(T^*_2 - T^*_1\right) \\
		\frac{dT^*_2}{dt^*} &= k_T\left(T^*_{02} - T^*_2\right) + \left|q^*\right|\left(T^*_1 - T^*_2\right) \\
		\frac{dS^*_1}{dt^*} &= k_S\left(S^*_{01} - S^*_1\right) + \left|q^*\right|\left(S^*_2 - S^*_1\right) \\
		\frac{dS^*_2}{dt^*} &= k_S\left(S^*_{02} - S^*_2\right) + \left|q^*\right|\left(S^*_1 - S^*_2\right)
	\end{align*}
	\end{footnotesize}
	
	\noindent Diese 4 Gleichungen werden auf 2 Gleichungen reduziert, indem man die Differenzen betrachtet
	\begin{footnotesize}
	\begin{align*}
		\frac{dT^*}{dt^*} = \frac{dT^*_1}{dt^*} - \frac{dT^*_2}{dt^*} &= 
		k_T\left(T^*_{0} + T^*_{02} - T^* - T^*_2 \right) + \left|q^*\right|\left(T^*_2 - T^*_1\right)  - \left( k_T\left(T^*_{02} - T^*_2\right) + \left|q^*\right|\left(T^*_1 - T^*_2\right) \right) \\
		&= k_T\left(T^*_{0} - T^*\right) + k_T\left(T^*_{02} - T^*_2\right) + \left|q^*\right|\left(T^*_2 - T^*_1\right) - k_T\left(T^*_{02} - T^*_2\right) - \left|q^*\right|\left(T^*_1 - T^*_2\right) \\
		&= k_T\left(T^*_{0} - T^*\right) - 2\left|q^*\right|\left(T^*_1 - T^*_2\right) \\
		&= k_T\left(T^*_{0} - T^*\right) - 2\left|q^*\right|T^*
	\end{align*}
	\end{footnotesize}
	
	\noindent Äquivalent wird dies für die Gleichungen des Salzgehaltes gemacht. Daraus erhalten wir nun die beiden Gleichungen:
	\begin{footnotesize}
	\begin{align*}
		\frac{dT}{dt} &= k_T\left(T_{0} - T\right) - 2\left|q\right|T \\		
		\frac{dS}{dt} &= k_S\left(S_{0} - S\right) - 2\left|q\right|S
	\end{align*}
	\end{footnotesize}

	
	
\end{document}