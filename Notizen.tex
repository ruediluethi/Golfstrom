\documentclass[11pt,a4paper]{article} 
\usepackage[utf8]{inputenc}
\usepackage[ngerman]{babel}
\usepackage{amsmath}
\usepackage{amssymb}
\usepackage{geometry}
\usepackage{hyperref}
\usepackage[table]{xcolor}
\geometry{
  left=3cm,
  right=3cm,
  top=3cm,
  bottom=3cm,
  bindingoffset=0mm
}
\usepackage{graphicx}
\usepackage{dsfont}

\begin{document}

	Bessere Aproximation für die Dichte-Funktion.
	
	Die Mathematische Modellbildung wie in 13.2 beschrieben implementieren und Schrittweise laufen lassen. Was sind sinnvolle Werte für \( T_0, S_0, k_t, k_s,  a, b, c\) ?

	(Erweitern des Modells auf n-Behälter?)

	Was bedeutet Skalierung?	 Das Verhältnis \( S = \frac{S^*}{S^*_0} \) hat keine Einheit(Dimension?) mehr. Ich gehe davon aus, dass zum Zeitpunkt \( t=0, S=1 \) gelten muss?
	
	Es wird stets mit \( |q| \) argumentiert, ist dies weil der Fluss jeweils in beide Richtungen geht? Das Model kann ja nur stabil sein, wenn auch wieder etwas zurück fließt?
	
	Simulieren des Models anhand historischer Werte: \\
	\url{https://data.giss.nasa.gov/gistemp/} (Temperatur) \\
	\url{https://podaac.jpl.nasa.gov/SeaSurfaceSalinity} (Salzgehalt) \\
	Gibt sogar ein Datenimport für MATLAB: \url{https://podaac.jpl.nasa.gov/dataset/SMAP_RSS_L3_SSS_SMI_MONTHLY_V2?ids=&values=&search=aquarius%20%2Bproject}
	
	\( T_{0 1/2} \) Jahrestemperaturschwankungen als Sinus...
	
	Wie viel Salz müsste ins Nordmeer entleert werden um den Fluss auf dem gleichen Niveau halten zu können?
	
	Auswirkungen auf das Klima? Schnittstelle zu einem Temperatur-Modell von Europa?
	
\end{document}